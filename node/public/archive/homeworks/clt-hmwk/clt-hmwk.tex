\documentclass[12pt]{article}

\parindent=.25in
\setlength{\oddsidemargin}{0pt}
\setlength{\textwidth}{440pt}
\setlength{\topmargin}{0in}

\usepackage{amsmath}
\usepackage{amsfonts}
\usepackage[dvips]{graphicx}
\usepackage{verbatim}
\usepackage{appendix}

\newcommand{\sign}{\mathrm{sign}}

\title{COMS 6253 Problem 1}
\author{Mengqi Zong $<mz2326@columbia.edu>$}

\begin{document}

\maketitle

\setlength{\parindent}{0in}

{\bf Problem 1.} The parity function on k 0/1-valued variables is

\begin{equation*}
PAR(x_1,...,x_k) = x_1 + ... + x_k \text { mod } 2,
\end{equation*}

a)  Show that the parity function on $\log s$ variables can be
computed by a decision tree of size $s$. \\

b)  Show that any PTF for the parity function on $k$ variables must
have degree at least k. \\

{\bf Answer:} \\

a) Since decision tree is an universal representation scheme, any
Boolean function with $k$ variables can be computed by a complete
decision tree with depth $k$ and $2^k$ leaves that exhaustively
queries all $k$ variables on every path. \\

For the parity function with $\log s$ variables, its complete decision
tree has depth $\log s$ with $2^{\log s} = s$ leaves. So the parity
function on $\log s$ variables can be computed by a decision tree of
size $s$. \\

b) We know from class that polynomial threshold function is an
universal representation scheme. So any Boolean function with $k$
variables can be computed by a PTF with degree at most $k$. As a
result, the parity function can also be represented by a PTF with
degree at most $k$. \\

Observed from $PAR(x_1,...,x_k) = x_1 + ... + x_k \text { mod } 2$,
we can see that one important property of parity function is that the
value of a parity function depends only on the sum of all $k$
variables. In other words, $PAR(x_1,  ..., x_k)$ is invariant under
permutation of its input. That is, $PAR(x_1,  ..., x_k)$ is a
symmetric polynomial. \\

One property of symmetric polynomial is that \emph {for any symmetric
  polynomial $p$ of degree $d$, there exists a polynomial $q$ of
  degree $d$ such that}

\begin{equation*}
p(x_1,...,x_n) = q(x_1 + ... + x_n)
\end{equation*}

Suppose $PAR(x_1,  ..., x_k)$ can be computed by a symmetric
polynomial threshold function $p(x_1,  ..., x_k)$ of degree $d$. Then
$PAR(x_1,  ..., x_k)$ can also be computed by an univariate polynomial
threshold function $q(x_1 + ... + x_k)$ of degree $d$. \\

Note that for all $x_1, ..., x_k \in \{0,1\}^k$, $x_1 + ... + x_k \in
\{0, 1, ..., k\}$. Since $q(x_1 + ... + x_k)$ compute the parity, we
have

\begin{eqnarray*}
q(0) < 0 \\
q(1) > 0 \\
q(2) < 0 \\
q(3) > 0 \\
\vdots \\
\end{eqnarray*}

So, in order to fit the $k+1$ input values, function $q$ must have at
least $k$ roots. As a result, we get $d \ge k$. \\

To sum up, any PTF for the parity function on $k$ variables must have
degree at least $k$.

\end{document} 
