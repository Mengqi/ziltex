\documentclass[12pt]{article}
\parindent=.25in

\setlength{\oddsidemargin}{0pt}
\setlength{\textwidth}{440pt}
\setlength{\topmargin}{0in}

\usepackage{amsmath}
\usepackage[dvips]{graphicx}
\usepackage{verbatim}
\usepackage{appendix}

\usepackage{amssymb}
\usepackage{amsfonts}
\usepackage{latexsym}
\usepackage[center]{subfigure}
\usepackage{epsfig}
\usepackage{hyperref}

\title{Stat 4201 Homework 11}
\author{Mengqi Zong $<mz2326@columbia.edu>$}

\begin{document}

\maketitle

% no paragraph indentation
\setlength{\parindent}{0in}

\section*{Question 1}

1. Here is the Cox porportional hazards model:

\begin{verbatim}
Call:
coxph(formula = Surv(time, status) ~ rx, data = colon.2)


             coef exp(coef) se(coef)      z      p
rxLev     -0.0266     0.974    0.110 -0.241 0.8100
rxLev+5FU -0.3717     0.690    0.119 -3.130 0.0017

Likelihood ratio test=12.2  on 2 df, p=0.0023  n= 929, number of events= 452
\end{verbatim}

The hazard rate for Levamisole relative to 5-FU is 0.690. \\

2. The value ``exp(coef)'' is the hazard rate of the specified
treatment. For example, ``exp(coef)'' 0.974 is the hazard rate for
treatment ``rxLev''. \\

The 95\% confidence interval for hazard rate is:

\begin{verbatim}
> exp(confint(fitcox.p1))
              2.5 %    97.5 %
rxLev     0.7844054 1.2087109
rxLev+5FU 0.5463673 0.8702657
\end{verbatim}

3. Form part 1, we can see that the p-value for Levamisole relative to
5FU is 0.0017. This means that the proportional hazards assumption for
Levamisole relative to 5FU is valid. \\

4. Here is the Cox proportional hazards model adjusting for Age and Sex:

\begin{verbatim}
Call:
coxph(formula = Surv(time, status) ~ rx + age + sex, data = colon.2)


               coef exp(coef) se(coef)        z      p
rxLev     -0.027482     0.973  0.11034 -0.24907 0.8000
rxLev+5FU -0.373819     0.688  0.11885 -3.14528 0.0017
age        0.002368     1.002  0.00405  0.58528 0.5600
sex       -0.000424     1.000  0.09431 -0.00449 1.0000

Likelihood ratio test=12.5  on 4 df, p=0.014  n= 929, number of events= 452
\end{verbatim}

\appendix
\appendixpage
\addappheadtotoc

The R code is listed below:

\verbatiminput{hmwk11.r}

\end{document} 
