\documentclass[12pt]{article}
\parindent=.25in

\setlength{\oddsidemargin}{0pt}
\setlength{\textwidth}{440pt}
\setlength{\topmargin}{0in}

\usepackage{amsmath}
\usepackage[dvips]{graphicx}
\usepackage{verbatim}
\usepackage{appendix}

\usepackage{amsthm}

\usepackage{amssymb}
\usepackage{amsfonts}
\usepackage{latexsym}
\usepackage[center]{subfigure}
\usepackage{epsfig}
\usepackage{hyperref}

\newtheorem{theorem}{Theorem}
\newtheorem{observation}[theorem]{Observation}
\newtheorem{example}[theorem]{Example}


\title{Stat 4201 Project: Is Carmelo Anthony helping the Knicks?}
\author{Mengqi Zong $<mz2326@columbia.edu>$}

\begin{document}

\maketitle

% no paragraph indentation
% \setlength{\parindent}{0in}

\section{Introduction}

How to measure the productivity of an individual participating in a
team sport? This has always been an interesting and important
question. \\

In this paper, we will explore methods of linking the player's
statistics in the National Basketball Association (NBA) to team
wins. Specifically, we will measure the productivity of New York 
Knicks player Carmelo Anthony to his team based on 18 consecutive
games of Knicks from February 6th 2012 to March 12th 2012.

\section{Background}

Carmelo Klyan Anthony is an American professional basketball player
who currently plays for the New York Knicks in NBA. Since entering the
NBA, Anthony has emerged as one of the most well-known and skilled
players in the league. Now, he is being considered as the superstar
of New York Knicks. \\

Despite his excellent basketball skills, Anthony has been long
criticized for not being able to bring wins to his team. The level of
the doubt becomes severe than ever after March 12th 2012, that Knicks
suffered a 6 straight lost. \\

On February 6th, New York Knicks player Jeremy Lin was promoted to the
starting lineup at that day's game of New York Knicks versus New
Jersey Nets. Since that game, due to Lin's excellent performance,
Knicks produced a winning streak. \\

However, it is worth pointing out that Carmelo Anthony got injured in
the February 6th's game and only played 5 minutes. Later, he missed
the next 7 games. During Anthony's absence, Knicks won 7 of the 8
games. However, after Anthony's return, Knicks only won 2 of 10 games
from Feb 20th to March 12th. All 18 game records are shown in
Table-\ref{tab:data1}.

\begin{table}[ht!]
  \begin{center}
    \begin{tabular}{|l|c|r|c|l|c|r|c|}
      \hline
Date      & Opponent & Scores  & W/L &
Date      & Opponent & Scores  & W/L \\ \hline
Feb. 6th  & UT       & 99-88   & W   &
Feb. 20th & NJ       & 92-100  & L   \\ \hline
Feb. 8th  & WAS      & 107-93  & W   &
Feb. 22nd & ATL      & 99-82   & W   \\ \hline
Feb. 10th & LAL      & 92-85   & W   &
Feb. 23rd & MIA      & 88-102  & L   \\ \hline
Feb. 11th & MIN      & 100-98  & W   &
Feb. 29th & CLE      & 120-103 & W   \\ \hline
Feb. 14th & TOR      & 90-87   & W   &
Mar. 4th  & BOS      & 111-115 & L   \\ \hline
Feb. 15th & SAC      & 100-85  & W   &
Mar. 6th  & DAL      & 85-89   & L   \\ \hline
Feb. 17th & NO       & 85-89   & L   &
Mar. 7th  & SA       & 105-118 & L   \\ \hline
Feb. 19th & DAL      & 104-97  & W   &
Mar. 9th  & MIL      & 114-119 & L   \\ \hline
          &          &         &     &
Mar. 11th & PHI      & 94-106  & L   \\ \hline
          &          &         &     &
Mar. 12th & CHI      & 99-104  & L   \\ \hline
    \end{tabular}
  \end{center}
  \caption{Brief game records \label{tab:data1}}
\end{table}

After Knicks's losing streak, the media blame Anthony for Knick's bad
performance. We will try to figure out if it is Anthony's fault.

\section{Model 1: Categorical Data Analysis}

\subsection{Data Processing}

The data can be displayed as a $2 \times 2$ table of counts, which
lists the numbers of games falling in each cross-classification of a
row factor (Anthony is absent or Anthony is available) and a column
factor (Knicks won or Knicks lost). The categorical data is shown in
Table-\ref{tab:cat}.

\begin{table}[ht!]
  \begin{center}
    \begin{tabular}{|l|c|c|}
      \hline
                            & Knicks Won & Knicks Lost  \\ \hline
      Anthony is absent     & 7          & 1            \\ \hline
      Anthony is available  & 2          & 8            \\ \hline
    \end{tabular}
  \end{center}
  \caption{Categorized game records \label{tab:cat}}
\end{table}

\subsection{Fisher's Exact Test}

We can apply Fisher's Exact Test to the data to test whether the
two probabilities of winning when Anthony is absent or not are the
same. Here is the output from R:

\begin{verbatim}
	Fisher's Exact Test for Count Data

data:  win.data 
p-value = 0.01522
alternative hypothesis: true odds ratio is not equal to 1 
95 percent confidence interval:
    1.533351 1396.017102 
sample estimates:
odds ratio 
  21.39941 
\end{verbatim}

As we can see, the p-value is 0.01522. This indicates that the two
probabilities of winning when Anthony is absent or not are not likely
to be the same. More specifically, when Anthony is available, the Knicks
tends to lose more.

\subsection{A closer look at this model}

There are certain things we have to consider about this model:

\begin{enumerate}
  \item The sample size maybe too small to make an accurate
    prediction.

  There are only 18 sample units in total. And the 95 percent
  confidence interval of the odds ratio is $[1.53, 1396.02]$. The
  confidence interval is way too wide, which means that the result is
  not accurate enough.
  \item The two probabilities of winning are different does not
    necessarily imply that this is all Carmelo Anthony's fault.

  Even if the Fisher's exact test is reliable, we still can't draw the
  conclusion that Knicks's bad performance is all Anthony's
  fault. Because there are many factors that affect the winning
  and losing of a basketball team we haven't considered in this model:

  \begin{itemize}
    \item The strength of the opponents.

      Apparently, the probability of winning when playing against
      weak opponents is higher than that of when playing against
      strong opponents. In this model, the underlying assumption is
      that the strength of the opponents in the two time periods are
      roughly the same.

    \item Home court advantage.

      It has been widely believed that the team playing at their home
      court has an advantage. In this model, the underlying assumption
      is there's no home court advantage out there.

    \item Teammates' performance.

      Basketball is a team sports, the performance of Anthony's
      teammates will surely affect the probability of winning and
      losing of the Knicks.

  \end{itemize}
\end{enumerate}

To sum up, this model is an easy and straight forward model. However,
it fails to consider many factors that affects the results of a
game.

\section{Observations}

A much detailed version of the game records is shown in
Table-\ref{tab:obs}. Note that whether the opponent is a strong or not
is based on my empirical experience. From the records, we draw the
follow observations. \\

\begin{observation}
This is only 2 of 8 strong opponents when Anthony is absent, but there
are 5 of 10 strong opponents when Anthony is back.
\end{observation}

This indicates that the strength of opponents for the two time period
is not the same.

\begin{observation}
When Anthony is absent, Knicks played against its 2 strong opponents
in New York with a home court advantage. When Anthony is avaible, they
played against its 5 strong oppoents out of New York without a home
court advantage.
\end{observation}

To sum up, observations show that we should take the strength of the
opponents and home court advantage into consideration to achieve more
accuarte result.

\begin{table}[ht!]
  \begin{center}
    \begin{tabular}{|l|c|r|l|l|c|l|c|r|l|l|c|}
      \hline
      Date      & Opponent & Scores  & Home & Strong OP & W/L \\ \hline
      Feb. 6th  & UT       & 99-88   & Home & False     & W   \\ \hline
      Feb. 8th  & WAS      & 107-93  & Away & False     & W   \\ \hline
      Feb. 10th & LAL      & 92-85   & Home & True      & W   \\ \hline
      Feb. 11th & MIN      & 100-98  & Away & False     & W   \\ \hline
      Feb. 14th & TOR      & 90-87   & Away & False     & W   \\ \hline
      Feb. 15th & SAC      & 100-85  & Home & False     & W   \\ \hline
      Feb. 17th & NO       & 85-89   & Home & False     & L   \\ \hline
      Feb. 19th & DAL      & 104-97  & Home & True      & W   \\ \hline
      \hline
      Feb. 20th & NJ       & 92-100  & Home & False     & L   \\ \hline
      Feb. 22nd & ATL      & 99-82   & Home & False     & W   \\ \hline
      Feb. 23rd & MIA      & 88-102  & Away & True      & L   \\ \hline
      Feb. 29th & CLE      & 120-103 & Home & False     & W   \\ \hline
      Mar. 4th  & BOS      & 111-115 & Away & True      & L   \\ \hline
      Mar. 6th  & DAL      & 85-89   & Away & True      & L   \\ \hline
      Mar. 7th  & SA       & 105-118 & Away & True      & L   \\ \hline
      Mar. 9th  & MIL      & 114-119 & Away & False     & L   \\ \hline
      Mar. 11th & PHI      & 94-106  & Home & False     & L   \\ \hline
      Mar. 12th & CHI      & 99-104  & Away & True      & L   \\ \hline
    \end{tabular}
  \end{center}
  \caption{A much detailed version of the game records \label{tab:obs}}
\end{table}

\section{Model 2: Logistic Regression}

\subsection{Data Processing}

The main problem of the data processing is how to measure the strength
of an oppoent. We will use the opponent's winning percentage till the
day of that game to indicate the strength of this opponent.

\begin{example}
Till Feb. 6th, Utah Jazz played 31 games and won 15 of them. The
winning percentage of Utah at Feb. 6th is
\begin{equation*}
p = 15 \div 31 = 0.48
\end{equation*}
\end{example}

One concern about this method is that each team's schedule is
different, so it is possible that the winning percentage defined can
not accurately indicate the strength of a team. However, we will show
that this is not a big problem. Till Feb. 6th, all teams have played
at least 30 games and there are 30 teams in the league. This means
that it is quite possible all teams have played against each other at
least once. So the schedule doesn't affect much here. \\

The data used in this model is shown in Table-\ref{tab:log}

\begin{table}[ht!]
  \begin{center}
    \begin{tabular}{|l|c|r|l|l|c|l|c|r|l|l|c|}
      \hline
      Number    & W/L & PCT   & Home & Anthony    \\ \hline
      1         & W   & 0.48  & Home & Absent     \\ \hline
      2         & W   & 0.22  & Away & Absent     \\ \hline
      3         & W   & 0.46  & Home & Absent     \\ \hline
      4         & W   & 0.44  & Away & Absent     \\ \hline
      5         & W   & 0.30  & Away & Absent     \\ \hline
      6         & W   & 0.28  & Home & Absent     \\ \hline
      7         & L   & 0.26  & Home & Absent     \\ \hline
      8         & W   & 0.60  & Home & Absent     \\ \hline
      \hline
      9         & L   & 0.32  & Home & Available  \\ \hline
      10        & W   & 0.55  & Home & Available  \\ \hline
      11        & L   & 0.82  & Away & Available  \\ \hline
      12        & W   & 0.38  & Home & Available  \\ \hline
      13        & L   & 0.57  & Away & Available  \\ \hline
      14        & L   & 0.61  & Away & Available  \\ \hline
      15        & L   & 0.73  & Away & Available  \\ \hline
      16        & L   & 0.44  & Away & Available  \\ \hline
      17        & L   & 0.58  & Home & Available  \\ \hline
      18        & L   & 0.75  & Away & Available  \\ \hline
    \end{tabular}
  \end{center}
  \caption{data for logit regression \label{tab:log}}
\end{table}

\subsection{The Logit Regression}

Let $\pi$ denote the Knicks' probability of winning. We build the
following logit regression model:

\begin{equation*}
logit(\pi) = \beta_0 + \beta_1 \cdot Home + \beta_2 \cdot PCT +
\beta_3 \cdot CA
\end{equation*}

In this model, Home is an indicator variable that denote if Knicks is
the home team. PCT is the opponents winning percentage.  CA is the
indicator variable that denote if Carmelo Anthony is available in this
game. \\

Here is the output from R:

\begin{verbatim}
Call:
glm(formula = Win ~ Home + PCT + CA)

Deviance Residuals: 
     Min        1Q    Median        3Q       Max  
-0.95772  -0.12527  -0.02737   0.18760   0.68669  

Coefficients:
            Estimate Std. Error t value Pr(>|t|)  
(Intercept)   0.7988     0.3489   2.289   0.0381 *
Home          0.1857     0.2023   0.918   0.3742  
PCT          -0.1045     0.6909  -0.151   0.8819  
CA           -0.6129     0.2333  -2.627   0.0199 *
---
Signif. codes:  0 ‘***’ 0.001 ‘**’ 0.01 ‘*’ 0.05 ‘.’ 0.1 ‘ ’ 1 

(Dispersion parameter for gaussian family taken to be 0.1651435)

    Null deviance: 4.500  on 17  degrees of freedom
Residual deviance: 2.312  on 14  degrees of freedom
AIC: 24.141

Number of Fisher Scoring iterations: 2
\end{verbatim} 

We can see that the p-value of CA is 0.0199, which indicates that CA
is an important variable in this model. Also, the drop-in-deviance
test gives the similar result:

\begin{verbatim}
Model:
Win ~ Home + PCT + CA
       Df Deviance    AIC scaled dev. Pr(>Chi)   
<none>      2.3120 24.141                        
Home    1   2.4511 23.193      1.0518 0.305099   
PCT     1   2.3158 22.171      0.0294 0.863845   
CA      1   3.4518 29.355      7.2140 0.007234 **
---
Signif. codes:  0 ‘***’ 0.001 ‘**’ 0.01 ‘*’ 0.05 ‘.’ 0.1 ‘ ’ 1 
\end{verbatim} 

So we get

\begin{equation*}
logit(\pi) = 0.7988 + 0.1857 \cdot Home - 0.1045 \cdot PCT -
0.6129 \cdot CA
\end{equation*}

Since the coefficient of CA is  $-0.6129$, this indicates that Carmelo
Anthony has a negative effect on Knicks. This means that when Carmelo
Anthony is available, Knicks tends to lose more.

\subsection{A closer look at this model}

Here are some comments about this model:

\begin{enumerate}
  \item The sample size may not be enough.

    Recall that in class, we learned that the rule of thumb for the
    sample size of a regression is 1 variable needs 10 sample
    units. Since there are 3 variables in the logit model, the ideal
    sample size should be at least 30. However, there are only 18
    sample units. This may lead to an inaccurate variable estimation.

  \item The data are not independent.

    The assumption of regression requires all data are independent
    from each other. However, the independent assumption doesn't hold
    here because in each season two teams will play against each other
    multiple times. For example, Knicks played against Dallas
    Mavericks at Feb. 19th. Then the two team played against each
    other again at Mar. 6th.

    We will show that this doesn't affect our model much since the
    only team that appeared multiple time is Dallas Mavericks for
    twice.

  \item Model team wins as binary responses may not be sufficient.
    
    Since each game's situation differs from each other, the binary
    responses may not be a good model to check if Anthony is helping
    the Knicks. Suppose Anthony is helping the Knicks, but the
    opponent is the second time period is too strong for Knicks to get
    a win. In this case, we will incorrectly draw the conclusion that
    Anthony has a negative effect on the team.

  \item Teammates' performance is not concerned in the model.

    We will show that this is not a big deal. Because it is rational
    to assume that all player's performance is consistent.

\end{enumerate}

To sum up, basically, the logit regression is a good model to show if
Carmelo Anthony is helping the Knicks.

\section{Conclusion}

Based on the results given by the two models, we draw the conclusion
that Carmelo Anthony is not helping the Knicks during Feb. 20th to
Mar. 12th.

\appendix
\appendixpage
\addappheadtotoc

The R code is listed below:

\verbatiminput{project.r}

\end{document} 
