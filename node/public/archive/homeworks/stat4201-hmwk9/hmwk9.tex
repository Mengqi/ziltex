\documentclass[12pt]{article}
\parindent=.25in

\setlength{\oddsidemargin}{0pt}
\setlength{\textwidth}{440pt}
\setlength{\topmargin}{0in}

\usepackage{amsmath}
\usepackage[dvips]{graphicx}
\usepackage{verbatim}
\usepackage{appendix}

\usepackage{amssymb}
\usepackage{amsfonts}
\usepackage{latexsym}
\usepackage[center]{subfigure}
\usepackage{epsfig}
\usepackage{hyperref}

\title{Stat 4201 Homework 9}
\author{Mengqi Zong $<mz2326@columbia.edu>$}

\begin{document}

\maketitle

% no paragraph indentation
\setlength{\parindent}{0in}

\section*{Question 1}

The number of mates in this in this population increased with the body
size. There is evidence that the distribution of number of mates in
this population is related to body size (The two-sided p-value for
bodysize is 0.002). The estimated mean number of mates for a 95 mm
body size male bullfrog is 0.0685, and the mean increased by a factor
of 1.77 for each 10-mm increase in body size up to about 150 mm (95\%
confidence interval for the multiplicative factor: 0.02332
0.0964). \\


The coefficients from the Poisson Log-Linear Model is as follow:

\begin{verbatim}
Coefficients:
            Estimate Std. Error z value Pr(>|z|)   
(Intercept) -8.11840    2.59380  -3.130  0.00175 **
bodysize     0.05723    0.01851   3.092  0.00199 **
\end{verbatim}

Here is the confidence interval:

\begin{verbatim}
                   2.5 %      97.5 %
(Intercept) -13.66943377 -3.43561440
bodysize      0.02332952  0.09636363
\end{verbatim}

Here is the Pearson Chi-Squared Goodness-of-Fit Test:

\begin{verbatim}
         Df Deviance Resid. Df Resid. Dev P(>|Chi|)    
NULL                        37     39.956              
bodysize  1   11.952        36     28.003 0.0005458 ***
---
Signif. codes:  0 ‘***’ 0.001 ‘**’ 0.01 ‘*’ 0.05 ‘.’ 0.1 ‘ ’ 1 
\end{verbatim}

\appendix
\appendixpage
\addappheadtotoc

The R code is listed below:

\verbatiminput{hmwk9.r}

\end{document} 
